% Chapter 1

\chapter{Introduction} % Main chapter title

\label{introduction} % For referencing the chapter elsewhere, use \ref{Chapter1} 

\lhead{Chapter 1. \emph{Introduction}} % This is for the header on each page - perhaps a shortened title

%----------------------------------------------------------------------------------------


Mobile Devices are rapidly evolving to very powerful personal devices full of sensors that provide several kind of context information, combined with great computing power in an hand held device. \\
This level of computing power was reserved for stationary desktop computers a few years ago, or to room filling supercomputers some decades ago.
At the same time, the devices are spreading out to more and more people.

Various sensors for measuring the have been miniaturized 

This development empowers a new kind of computing to find it's way out of research laboratories, where such systems ran on special hardware in artificial environments, into our daily lives: \cite[cf.][]{age-of-context}
\emph{context-aware computing}. 


\section{The goal of this work}

This thesis is about designing and implementing a context-aware mobile system for guiding a visitor through indoor and outdoor exhibitions like parks and a museums.

The goal is to design a flexible software platform that is easily adaptable to fit different exhibitions and runs on commodity hardware. 

Several analytic functions 

The system consists of two main parts:

The first one is a backend system that allows to define all relevant textual and audio information describing the exhibitions areas and single exhibits using a web interface and to process and visualize collected data.

The second one is the application running on the mobile device of the visitor. The goal is to provide relevant information at the right time to the visitor, thus enhancing his experience, and to collect anonymized sensor data for later analytics.

This thesis is also an experiment about moving on from established languages and platforms to emerging ones: From C-like languages and Java to Swift and Scala, from relational databases to NoSQL, from the Java web architecture to Play, from NFC to Bluetooth low energy beacons.

\section{Motivation}

This work is part of the foundation for the first product of the start-up contexagon GmbH in Kreuzlingen, Switzerland. The company was founded in September 2014 by the author, Maurizio Tidei, and Sascha Lorenz \footnote{S. Lorenz is currently writing his Master Thesis on Big Data Processing with Apache Spark, e.g. for analyzing visitor interests} after several conversations with museum directors in the tourist region of the Lake Constance in Summer 2014, who expressed interest in such a solution and encouraged us in pursuing the development of an affordable system based on commodity hardware and a highly configurable software platform.

The first project is a context-aware guide for the World War I exhibition at the Schloss Arenenberg in Salenstein, that is scheduled for August 2015. Contexagon will realize the technical part of the exposition in cooperation with Steiner Sarnen Schweiz AG , a company designing and realizing exhibitions and other visitor attractions since 1997 mainly in Switzerland, Germany, Austria and Italy \cite{steinersarnen}.

\section{The Approach}

Chapter 2 handles the theoretical background and fundamentals of this work. It starts with a definition, categorization and the history of context-aware computing.
Since the location context is key for the system that has to be designed, different methods for providing and defining the location context are evaluated. For the evaluation of some new technologies, two technology prototypes are written and described in this chapter, too.

The context-aware guide system overall architecture and design will then be discussed in Chapter 3.

In Chapter 4, the design and implementation of the front end, the guide's mobile application written in Swift, is described.

Chapters 5 addresses the system's back end used for modeling the indoor or outdoor site, e.g. an exhibition or park guide, and for running analytics on anonymous visitor movement paths and feedback for the optimization of visitor flows and visitor experience. This part of the system is designed as a web application in Scala and Play, with Couchbase NoSQL 

Chapter 6 handles the conclusion and future work. 