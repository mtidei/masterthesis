% Chapter 3

\chapter{System Overview} % Main chapter title

\label{systemoverview} % For referencing the chapter elsewhere, use \ref{Chapter1} 

\lhead{Chapter 3. \emph{System Overview}} % This is for the header on each page - perhaps a shortened title

%----------------------------------------------------------------------------------------

\section{Introduction}

The Context-Aware Guide (subsequently called "CA Guide") needs to be defined to fit indoor and outdoor exhibitions, like classical museums, parks and gardens. 
The basic functionality must be accessible even by persons not familiar with mobile technology. The context awareness techniques can help avoid big parts of explicit user input.

The following Image shows the complete system, consisting of a museum guide frontend running on a mobile device, a backend for the configuration of the guide and analytic function and a database server.

Image Guide Frontend, Backend and Database Server

Image Mockup CA Guide 

Image A Backend Screen Mockup

\section{Data Structure}

\subsection{Exhibition Modelling}

A single exhibition is modeled in the JSON data format. Compared to XML, JSON is more readable and easier to produce without dedicated tools in a simple text editor.

The entities needed for modelling an exhibition are:

Image entyty relationship diagram

With a normal relational database, pretty much of the entities would be saved in separate tables using primary keys and foreign keys to represent the relationships between the tables.

In a document-oriented database the whole exhibition can be stored in an aggregated way. 

\begin{lstlisting}
key: "exhibition-CXN01"
value: {
 "type":"exhibition",
 "city":"Konstanz",
 "pois": [{
	 "name":"Dom", 
	 "locationDefinition": {
	 	"type": "circle",
	 	"location": [15.0020312,2.3434322],
	 	"radius": 30,
	 	"floor": 1
	 }, 
	 "text":"", 
	 "images":[], 
	 "audio":{
	 	"level1":
	 	"level2":
	 	"level3":
	 }, 
	 "videos":[]},]
}
\end{lstlisting}

A JSON Schema is used for asserting a valid structure of a JSON file, similar to a DTD (Data Type Definition) in XML (http://json-schema.org).

\subsection{Sensor Data}

Sensor data is saved to the Couchbase database to allow replaying it for development, testing, presentations and for big data analytics presented in the system's web backend. 

The single measurements can be very high frequent, for example in case of accelerometer data, that is updated every 5 millisecons. New beacon measurements are available every second. Events on a higher logical level, like "entering region a", will occur much less frequent.

T


