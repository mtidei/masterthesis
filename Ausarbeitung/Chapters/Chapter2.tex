% Chapter 2

\chapter{Theoretical Background}

\lhead{Chapter 2. \emph{Theoretical Background}} % This is for the header on each page - perhaps a shortened title

%----------------------------------------------------------------------------------------
\section{What is Context-Aware Computing?}

The term "context-aware computing" was first used by B. Schilit et. al. 1994 in the paper "Context-Aware Computing Applications" \cite{Schilit94context-awarecomputing}. They introduce a this new kind of computing as a "new class of applications that are aware of the context in which they are run. Such context-aware systems adapts according to the location of use, the collection of nearby people, hosts, and accessible devices, as well as to changes to such things over time".

\subsection{Definition of Context}

Schilit defined context by enumeration of examples that are all related to location and proximity.

Dey and Abowd provided a precise universal definition of context in their paper "Towards a Better Understanding of Context and Context-Awareness" published 1999 \cite{Abowd99}: "Context is any information that can be used to characterize the situation of an entity. An entity is a person, place, or object that is considered relevant to the interaction between a user and an application, including the user and applications themselves."

While being precise, this definition is too general to get an idea of which kind of context information an application could use. So the following subchapter provides a wider list of context examples and a categorization.


\subsection{Categories of Context}

There are many different types of context: 

\subsubsection*{Computing Context}

Some examples are network connectivity, ommunication costs and bandwidth or available resources like printers, displays and input devices.

\subsubsection*{User Context}

\begin{itemize}
\item The user's \emph{identity}, including his age, gender, education, profession, interests. In today's age of social media the user's connections to other people and their identity are important and widely available as well.
\item The \emph{location} is the most obvious kind of user context. It can be subdivided in two different sub types of location: \\
  Logical location, e.g. "at home" or "at work" or slightly different "in a library, restaurant, cinema"\\
  Absolute location, e.g. "47\textdegree40'02.0"N 9\textdegree10'19.7"E" as geographic coordinates on earth or 2nd floor room 205 in the F-Building of HTWG Konstanz.
\item The \emph{emotional state} indicating if the user is happy, angry, worried etc. At which level is the personal stress level?
\item \emph{Fisical state}: Is the user rested or tired?
\item The user's \emph{health state}, determined by measuring some vital parameters (e.g. hearth rate, blood pressure and oxygen saturation). Many  dedicated low-cost hardware items for this kind of measurements were released in 2014 or are announced for this year. 
\item Information about other people or mobile devices nearby forms the \emph{co-location} context.
\item The current \emph{activity}. For example walking, driving a car, riding a bicycle or jogging. Some of this activities need to be further specificated, e.g. jogging to practice sport or jogging for catching a train.
\end{itemize}

\subsubsection*{Time Context}

The time context, such as the current time of the day, the day of the week, the current season or the full date. 

\subsubsection*{Physical Context}

Some examples for environment information are the current temperature, the weather, the intensity of ambient light, the noise level and even traffic condition.

cf. \cite{distributed-cas}


\subsection{Categories of Context-Aware Applications}

Schilit and his co-authors describe the following four categories of context-aware applications:

\begin{itemize}
\item Proximate Selection

Highlighting actions or information based on the current location of the user is called "Proximate Selection". While this user interface technique generally requires a user entering his location manually, context-aware systems default it automatically to the currently sensed location. \\
Nowadays, we encounter this behavior in many smartphone applications like weather forecasts for the current city, searching nearby stores in digital yellow-pages or even when performing online searches on non-portable computers.

\item Automatic Contextual Reconfiguration

"Automatic Contextual Reconfiguration" means loading and activating different system configurations based on the current context of use. For example, loading a different digital whiteboard per room gives the illusion of accessing it as if it was physically mounted in that room. But the considered context information is not limited to the location. Changing the energy plan of a notebook based on the connection status of the A/C power cable or the current battery level are further examples for this category. 

\item Contextual Information and Commands

This category contains systems providing the right piece of information and offering the adequate actions at the right time fitting the current context.

Retrieving information provided as text, audio, picture and video form, fitting to the current location context of the user is obviously a key feature of the museum guide that is developed as part of this master thesis.

\item Context-Triggered Actions

Context-triggered actions are applications that execute a defined action when a specific predefined context-state is reached. In contrast to contextual information and commands, these actions are automatically executed. Combining multiple rules allows designing more complex behaviors.

\end{itemize}




Obviously, nowadays there are many applications that fit in two or more categories. A modern smartphone based navigating system for example offers functions like "Take me to a gas station", displaying a list with the nearest one already preselected (proximate selection). When light conditions change, like when entering a tunnel or when it gets dark, the display is automatically dimmed. The pedestrian mode is an example for two categories: The device senses steps and switches automatically into pedestrian mode considering paths not accessible by car (automatic contextual reconfiguration) and offering commands like "Take me back to my parking lot" (contextual commands).



\subsection{Ubiquitous Computing}

The vision and research field of ubiquitous computing was coined 1991 by Mark Weiser of Xerox PARC in his article "The Computer for the 21st Century" \cite{weiserm1991}. 
Weiser's starts his article with these words:
"The most profound technologies are those that disappear. They weave themselves into the fabric of everyday life until they are indistinguishable from it."

For computers that means beeing integrated in common objects of daily usage, supporting humans without demanding their full attention. \\
As very good example for a technology that disappeared in the background he uses the vanishing of electric motors. At the beginning, big powerful engines powered a whole fabric, with axles and belts distributing the force to the single machines. Later each machine had its own motor and today several electric motors - miniaturized and bigger ones - are integrated into a single device. Often they work in the background letting us focus on the main task. Already in 1991, as he writes, a typical car had more than 20 motors to start the engine, lock and unlock the doors, clean the windshield, open and close the windows and so on. 

Although Weiser didn't use the term context-awareness, he already recognized the importance of the location context: "We have found two issues of crucial importance: location and scale. Little is more basic to human perception than physical juxtaposition, and so ubiquitous computers must know where they are. [...] If a computer merely knows what room it is in, it can adapt its behavior in significant ways without requiring even a hint of artificial intelligence.".

\section{Sensors of Mobile Devices}

\subsection{Accelerometer}

\subsection{Gyroscope}

Pedometer ref.

\subsection{Magnetometer}

\subsection{Barometer}

Some of the latest smartphones and tablets are equipped with a barometer for measuring the air pressure. This information combined with the current weather dependent ground pressure can be used to determine the current altitude or the floor of the building the device is in. 

\subsection{Proximity Sensor}

The proximity sensor typically measures the distance of the device's front to the next object. It is used to turn off the display when holding a mobile phone at the ear or to determine if the phone is inside a pocket.

\subsection{Light Sensor}

The light sensor measures the intensity of the light. It is primarily used to adapt the screen brightness to the ambient light.

\section{Localization Techniques}



\section{Bluetooth Low Energy iBeacons}

Bluetooth Low Energy (BLE) is a standard %TODO year

\section{Swift}

\subsection{What is Swift?}

Swift is a modern programming language released by Apple in 2014. In \cite{swift-book} Apple introduces Swift as "a new programming language for iOS and OS X that builds on the best of C and Objective-C, without the constraints of C compatibility".

Since Swift is a cutting-edge language, this section is dedicated more attention as one would normally do for a implementation language inside a thesis. 

Developers already familiar with the well designed functional programming language Scala will recognize several concepts. It has a concise Syntax avoiding big parts of boilerplate code and syntactic noise, supports functional programming and is statically typed.

The following section is an overview comparison of Scala and Swift features, created during the familiarization with Swift for this thesis. Language features marked will a * will be discussed separately in section \ref{majorDiffs}.

\subsection{Comparison Scala and Swift}

\newcommand{\yes}{yes}

\begin{longtable}{|P{5.5cm}|P{4cm}|P{4cm}|}

\hline \textbf{Language Feature} & \textbf{Scala} & \textbf{Swift} \endhead
\hline Type inference & \yes & \yes \\
\hline Line end separates commands (no need for semicolon) & \yes & \yes\\
\hline Implicit type conversions * & \yes & no \\
\hline Default access levels (access level has to be only provided if it differs from default) & \yes & \yes \\
\hline Functions are first class types\footnote{Functions can be passed to functions, returned from functions, created at runtime and assigned to variables} & \yes & \yes \\
\hline Closures & \yes & \yes \\
\hline Curried functions & \yes & \yes \\
\hline Operator functions & \yes & \yes \\
\hline Optionals * & via \smalltt{Option[T]} class & widely used, dedicated Syntax\\
\hline Switch with pattern matching & \yes & \yes \\
\hline String interpolation & \yes \newline \smalltt{"Hello \$nameVar"} & \yes \newline \smalltt{"Hello \textbackslash(nameVar)"} \\
\hline Keyword for variable definition & var radius = 1 & var radius = 1 \\
\hline Keyword for constant definition & let pi = 3.14 & val pi = 3.14 \\
\hline Array literal & \smalltt{Array(1,2,3)} & \smalltt{[1,2,3]} \\
\hline Map literal & \smalltt{Map(1->"a", 2->"b")} & \smalltt{[1:"a", 2:"b"]} \footnote{Maps are called dictionaries in Swift} \\
\hline If condition must be boolean & \yes & \yes \\
\hline Tuples & \yes, but without named elements & \yes\\
\hline Ranges & \smalltt{for i <- 0 to 4} \newline \smalltt{0 until 4} & \smalltt{for i in 0...4} \newline \smalltt{0..<4}\\
\hline Constructor & def this() & init() \\
\hline Extended getter/setter concept * & \yes, no observers & \yes, very flexible concept including observers (willSet() didSet() events) \\
\hline Interfaces & trait & protocol \\
\hline Extension of existing types & \yes & \yes \\
\hline Struct & no & \yes \\
\hline Enum & via extending the Enumeration class & \yes, dedicated keyword \\
\hline "Any" Type & Any, AnyVal, AnyRef & Any (instance of any type, even function types) \newline AnyObject (instance of any class type) \\
\hline Qeury an instance of a type by a key in brackets, like arrays or maps & obj(index) calls obj.apply method\newline obj(index) = newValue calls obj.update(0, newValue) & \smalltt{subscript(i:T) -> T2 \{ \newline
\phantom{.} get \{\ldots\} \newline
\phantom{.} set(newValue) \{\ldots\} \newline
\}
} \\
\hline Memory Management & JVM Garbage Collection & Automatic Reference Counting \\
\hline Nested Functions & \yes & \yes \\
\hline Generics & \yes & \yes \\
\hline 

%TODO static

\end{longtable}

\subsection{Major Differences to Scala} \label{majorDiffs}

In this section, the major differences between Scala and Swift that are encountered during the first Swift projects are described.

\subsubsection{Implicit Type Conversions}

In Swift, every type conversion has to be explicit. Even when using different numerical types inside an arithmetic expression, the conversion is not done automatically, in contrast to Scala and even Java. So this code yields a compile time error for example:

\begin{lstlisting}[frame=none]
let a = 1.0
let b = 2
let c = a + b // compiler error: cannot invoke '+' with an argument list of type '(@lvalue Double, @lvalue Int)
\end{lstlisting}

To get an integer 3 assigned to the variable 'c', you need to cast 'a' to Int. For a decimal 3.0, the variable 'b' has to be cast to Double.

\begin{lstlisting}[frame=none]
let c = Int(a) + b    // ok, c is 3 Int
let d = a + Double(b) // ok, d is 3.0 Double
\end{lstlisting}

Although initially it can be a bit frustrating running into compile errors of this kind, you get used to explicitly casting the values to the desired types fast. The advantage of this approach is that the developer explicitly sees the type of the resulting value without having to remember language specific rules.

In contrast, Scala has a powerful implicit type conversion system. Combined with operator overloading it enables developers to create beautiful internal DSL (Domain Specific Languages) that read more like natural language. \footnote{A good example for using implicit conversion to build a DSL query language can be found at \cite{scala-dsl-example}}
But, as M. Odersky rightly wrote in \cite[Chapter 6.13]{scala-book}, "... bear in mind that with power comes responsibility. If used unartfully, both operator methods and implicit conversions can give rise to client code that is hard to read and understand.".

\subsubsection{Optionals}

Variables are non nullable by default in Swift. So the following code will not compile:

\begin{lstlisting}[frame=none, language=swift]
var name = "Swift"
name = nil // compile time error: Type 'String' does not conform to protocol 'NilLiteralConvertible'
\end{lstlisting}

As a consequence, the variable can safely be accessed at any time without the danger of a NullPointerException respectively a nil runtime error.

To allow a variable to assume the value of nil (the null equivalent in Swift), it's type has to be defined as optional by the '?' postfix to the type name.

\begin{lstlisting}[frame=none]
var name:String? = "Swift"
println(name)  // prints 'Optional("Swift")' to the console
println(name!) // prints 'Swift'
name = nil     // ok
let statement = name + " is great" // compile time error: value of optional type 'String?' not unwrapped
println(name!) // runtime error: unexpectedly found nil while unwrapping an Optional value
\end{lstlisting}

The first println statement outputs 'Optional("Swift")' because the string value is wrapped inside the optional and needs to be unwrapped using the '!' postfix. Note that trying to unwrap an optional without value yields a runtime error.

A convenient way of querying properties or calling methods on optionals is called optional chaining. Imagine a circle object with an optional custom style that may have a border, which in turn has an optional custom color. To check the existence of the custom border color, instead of

\begin{lstlisting}[frame=none]
if circle.style != nil && circle.style.border != nil && circle.style.border.color != nil
\end{lstlisting}

it is possible to write

\begin{lstlisting}[frame=none]
if circle.style?.border?.color != nil
\end{lstlisting}

If any link in this chain is nil, the whole chain fails gracefully and returns nil.

Another concept in the context of optionals are failable initializers (known as constructors in other languages). When explicitly defining an initializer as failable appending a question mark (init?), it's return type is optional variant of the type it should initialize. That can be useful for handling invalid parameter values or other initialization problems.

Optionals are widely used in Swift's iOS APIs and their syntax one of the first things noticed when looking to Swift sources as a novice.  
In my opinion, the usage of optionals results in beeing more aware of the presence or absence of values and coding more prudently. Of course, optionals can only add real value if not blindly unwrapped just to silence the compiler errors.  

%TODO cite null pointer ideator

\subsubsection{Getter and Setter}

Swift has a well designed flexible property system, with a concise getter and setter syntax thanks to built-in language support. 

It addresses the two main problems, encapsulation and computing bound to accessing or mutating a property, classical getter and setter methods solve, without the overhead of writing separate accessing and mutating methods for an object's properties.

Encapsulation can be archived by restricting write access to a stored property with the private(set) keyword.

\begin{lstlisting}[frame=none]
private(set) var age = 55
\end{lstlisting}

This restricts write access to the current source file in Swift\footnote{It is still possible to restrict the access to instances of the class by using a separate file for the class definition.}.

In case some computation is needed to update dependent values, Swift offers the willSet and didSet observers that are executed right before and after a stored property is mutated.

\begin{lstlisting}[frame=none]
var age = 55 {
  didSet {
  }
  willSet {
  }
}
\end{lstlisting}

If a property is purely computed, it can be defined in a similar way.

\begin{lstlisting}[frame=none]
var name:String = {
  get {
  }
  set(newName) {
  }
}
\end{lstlisting}

The setter is optional. Nothing changes for the calls for accessing and mutating that property: obj.name = "newName" executes the set block, obj.name the get block.

\subsubsection{Exception Handling}

The exception concept is completely missing in Swift. Unfortunately, there is no official explanation from Swift's designers for this decision.

Using Swift's tuples it is possible to return multiple return values, for example an optional return value paired with an optional error object (cf. \citep{swift-error-handling}):
\begin{lstlisting}[frame=none]
func doSomething(param: String) -> (return: String?, error: NSError?)
\end{lstlisting}

Another option to consider is creating an enumeration with a success and a error member.

\begin{lstlisting}[frame=none]
enum Result {
    case Success(res:Int)
    case Error(msg:String)
}

func next() -> Result {
    return Result.Success(res:3)
}

switch next() {
case .Error(let msg):
    println("Something went wrong: " + msg)
case .Success(let res):
    println("OK: " + String(res))
}\end{lstlisting}

These alternatives misses the exceptions ability to retract some levels of the call stack and retry without explicitly passing an error object up the call chain.
%TODO Erfahrungen Fehlerbehandlung

\subsubsection{Summary}

Swift is currently the most modern programming language available for a mobile platform. It is a great step forward compared with it's predecessor Objective C. 
Althoung it's expressive and concise syntax and it's Playground and REPL (Read-Eval-Print-Loop) feature, Swift remains a compiled language. Like Objective C, it is compiled into machine code by the LLVM compiler, which proved to be very fast. 


%TODO Parallel execution

\section{iOS and Cocoa Touch}

iOS Cocoa Touch Framework 
Developing with XCode