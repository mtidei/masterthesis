% Chapter 5

\chapter{CA Guide Back End} % Main chapter title

\label{backend} % For referencing the chapter elsewhere, use \ref{Chapter1} 

\lhead{Chapter 5. \emph{CA Guide Back End}} % This is for the header on each page - perhaps a shortened title

%----------------------------------------------------------------------------------------

\section{Introduction}

The CA Guide back end is meant to be used by the park or museum staff responsible for designing and optimizing the visiting experience. In contrast to the front end, the user interface can be more complex requiring an introductory training and support. Of course, the usability deserves special attention in spite of the training, as it has a major influence on the frequency a software is used, the attitude towards it and so the success of the whole product. %TODO Citation

There are two main tasks accomplished with the CA Guide back end:

\begin{itemize}
\item Design of the Park/Museum
\item Analytics
\end{itemize}

%TODO Mockups
%TODO In Place editing

\section{Target Platform}

The back end user interface has to run in a standard web browser without specific plugins. This has several advantages: Users can start immediately to work with the product without installing any client. With the project data in a database in the cloud, they can work on the same project from different locations using several different computers without manually setting up a synchronization infrastructure. Computation expensive analytic functions can be performed directly on the database or web/application servers and only the results are transferred to the client, enabling it's usage on  hardware.

By allowing collaborative editing, the museum staff can easily be supported remotely in real time without having to be on site. 


\section{Setting up the Development Platform}

\subsection{Scala and Play}

Scala is used as programming language for the back end, with the Play web framework is used as a basis for developing a web application in Scala.

%less as css

\subsection{Adding Reactive Couchbase as Scala Database Driver}

%build.sbt
%connecting to couchbase database server (conf/applicatin.conf)
%conf/play.plugins

\subsection{Adding TypeScript for Web Client Code}

%Motivation for Typescript

\subsection{Adding Bootstrap as UI Framework}


\section{Architecture}

\subsection{Overview}

%Single page,  

\subsection{The REST HTTP Interface}

%HTTP methods
%Excerp routing file, reverse routing, single place of definition

\subsection{Performing Database Queries using Scala Futures}













